\documentclass[11pt]{article}
\usepackage{hyperref}
\usepackage[utf8]{inputenc}
\pagestyle{headings}
\usepackage{listings}

\begin{document}
\lstset{language=Java} 
\title{INF3470 - Digital signalbehandling \\ Høst 2015 - Semester}
\author{Hans Rusten}
\date{\today}
\maketitle

\section{Om faget}


Signalbehandling kan defineres som det matematiske verktøyet som brukes for å analysere, modellere og utføre operasjoner på fysiske signaler og deres kilder. I digital signalbehandling anvendes verktøyene på signaler slik de er representert i en datamaskin. Signalbehandling står sentralt i en rekke anvendelser innenfor bl.a.: trådløse kommunikasjonssystemer (mobiltelefoner og WLAN), medisin (ultralyd), seismikk, sonar, multimedia (f.eks. MP3), måleteknikk, og fjernanalyse.

Emnet behandler følgende tema: Analoge og digitale signaler og systemer i tids- og frekvensrommet, sampling og Z-transform, frekvenstransformasjoner og Fast Fourier-transform, og filtrering, FIR og IIR filtre.\\


Målet med kurset er at studentene skal:
\begin{itemize}

\item Lære å beherske basismetodene for behandling av både analoge og diskrete/digitale signaler og systemer.
\item Opparbeide en god teoretisk forståelse for representasjon av signaler i tids- og frekvensdomenet.
\item Kjenne til hvordan gjennomgått teori anvendes i viktige systemer som mobiltelefoner, lydkompresjon og avbildningssystemer.
\item Få en god bakgrunn for videre studier i signalbehandling.
\end{itemize}

\section{Komplese tall}
$$e^{j\phi} = cos(\phi) + sin(\phi)$$
Ut  ifra dette kan vi utlede eulers idenitet lett:
$$(e^{j\lambda}+e^{-j\lambda})/2 = cos(\lambda)$$ 
$$(e^{j\lambda}-e^{-j\lambda})/2 = j*sin(\lambda)$$ 
Generell tidsavhenning funksjon:
$$Acos(2\pi ft+\phi) = \frac{A(e^{j\pi ft}e^{j\phi}+e^{-j\pi ft}e^{-j\phi})}{2}$$
En cosinus med frekvens $f$ og fase $\phi$ kan tolkes
som summen av to komplekse eksponentiale.\\
Modulasjon:
$$cos(\phi_1)cos(\phi_2) = \frac{cos(\phi_1 + \phi_2) + cos(\phi_1 - \phi_2)}{2} $$
Geometriske rekker: 
\section{Diskret signaler og systemer}
Begreper: Left-side, Right-side, causal, anti-causal, enhetspuls, enhets-stegfunksjon,
\\
Egenskaper ved impulsfunksjonen:
$$x[n]\delta[n-k] = x[k]\delta[n-k] $$
Impulsen er bare $1$ ved $k = 0$
Tidsdiskrete sinuser:
$$ (co)sinus x(t) = cos(2\pi ft) $$
Samples ved $t_s$ der samplingsfrekvensen er $S = 1/{t_s}Hz$ dvs. $t = n*t_s$
Gir diskret cosinus:
$$ x[n] = cos(2\pi fnt_s) = cos(2\pi f/S) = cos(2\pi nF) $$
4 frekvensbegreper: 
\end{document}
